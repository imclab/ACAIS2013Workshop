The second object we will be adding is food. This gives our agents something that they can interact with.

\instruct{Create a new file food.py}

As we are going to use functions (and classes) of the breve engine, we must import it at the top of our file.

\begin{lstlisting}[language=Python]
import breve
\end{lstlisting}

\instruct{Add the import statement to the top of our agents file.}

\section{Creating the food object}

Our initial version of the food is very simple, it will simply stay on the field. Every food source must extend the \textit{breve.Mobile} class. Food does not have to move, but we want to be able to interact with it.

Just as with our controller and agent, we have to add a constructor and an \textit{iterate} function.

\begin{lstlisting}[language=Python]
class SimpleFood (breve.Mobile):

    def __init__(self):
        breve.Mobile.__init__(self)

    def iterate(self):
        None
\end{lstlisting}

\instruct{Add the class definition above to the agents file.}

From the previous section we know that this is not enough. We will also have to add the food to the controller

\begin{lstlisting}[language=Python]
self.SimpleFood = food.SimpleFood()
\end{lstlisting}

\instruct{Add the snippet to the \textit{\_\_init\_\_} function of the controller, just below the lighting settings.}

The food should now have been added to the simulation.

\instruct{Try to run the simulation and see what happens.}
As you can see you have now added a single food object to the field, though it does nothing yet.

\section{Location}
Our food is always at the same location when running the program. We want it to be more random. This can be done with the following function.
\begin{fullwidth}
\begin{lstlisting}[language=Python]
def randomizedLocation(self):
    randomLoc = breve.randomExpression(2 * breve.vector(20,20,20)) - breve.vector(20,20,20)
    self.move(randomLoc)
\end{lstlisting}
\end{fullwidth}
\instruct{Add the function above to the food class}

Now we just have to call this function when we create the food.
\begin{lstlisting}[language=Python]
self.randomizedLocation()
\end{lstlisting}

\instruct{Add the line above to the \textit{\_\_init\_\_} function of the food}

\section{Appearance}
We will again define one shape for all food sources.

\instruct{Add the following to the constructor of the SimulationController}
\begin{fullwidth}
\begin{lstlisting}[language=Python]
# Create a shape for the food sources
self.foodShape = breve.createInstances(breve.Sphere, 1).initWith(0.5)
\end{lstlisting}
\end{fullwidth}
This will create a small sphere. To be able to use this shape in our agent we add a getter method to the controller.

\instruct{Add the following getter to the simulation controller:}
\begin{lstlisting}[language=Python]
def getFoodShape(self):
    return self.foodShape
\end{lstlisting}

Now it is very easy to change the shape of the food. We change it with the following line of code:
\begin{lstlisting}[language=Python]
# Set the shape of the food source
self.setShape(self.controller.getFoodShape())
\end{lstlisting}
\instruct{Try to change the shape of the food by adding the code to the constructor of the food. Test it by running the simulation, has the shape changed?}

\section{Creating more food}
A single food object is not much for all our agents, so we want to add some more. We first added a single food object to the SimulationController with the following code:
\begin{lstlisting}[language=Python]
self.SimpleFood = food.SimpleFood()
\end{lstlisting}
We now want more food objects, this can be done with the following code:
\begin{lstlisting}[language=Python]
breve.createInstances(food.SimpleFood, 3)
\end{lstlisting}
\instruct{Replace the first piece of code with the second piece of code to create more food objects}