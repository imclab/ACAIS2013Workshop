% TODO:
% -Iemand anders Mac checken of je geen paden hoeft te zetten-
% Paden hoef je dus niet te zetten mits je in de breve folder zit en vanuit daar runt
% Demo aanpassen
% Eventueel standaard paden voorstellen

The program essential to this workshop is Breve. In this guide we will be guiding you through all necessary steps to get it up and running. 

\section{Windows}
	In this section we will explain how to set up Breve under Windows.

\subsection{Installation}
	Extract the files to a location of your choosing

\subsection{Previous Python installation}
	If you have previously installed Python and added this to the path, you will have to remove the path. You can leave Python installed. If you do not remove the path, Breve will try to use the wrong Python version. Where you can find these paths can be seen in the following section.

\subsection{Setting paths}
	You will have to set two paths. One for Breve and one for the Python version that comes with Breve.

	There are two possibilities here. The first one is temporary and has to be run in command prompt every time you start a new command prompt session. It consists of the following two lines:
	\begin{verbatim}
		  set BREVE_CLASS_PATH=<breve_path>\lib\classes
		  set PYTHONPATH=\%PYTHONPATH\%;<breve_path>\lib\python2.3
	\end{verbatim}
%	\begin{lstlisting}
%		set BREVE_CLASS_PATH=<breve_path>\lib\classes
%		set PYTHONPATH=\%PYTHONPATH\%;<breve_path>\lib\python2.3
%	\end{lstlisting}
	If, for example, you have copied Breve to `C:\\breve\_2.7.2' it would look like this:
	\begin{verbatim}
		  set BREVE_CLASS_PATH=C:\breve_2.7.2\lib\classes
		  set PYTHONPATH=\%PYTHONPATH\%;C:\breve_2.7.2\lib\python2.3
	\end{verbatim}
%	\begin{lstlisting}
%		set BREVE_CLASS_PATH=C:\breve_2.7.2\lib\classes
%		set PYTHONPATH=\%PYTHONPATH\%;C:\breve_2.7.2\lib\python2.3
%	\end{lstlisting}

	There is an alternative version which is more permanent, but can easily be undone once it is not required anymore. This is done by changing the environment veriables in the Advanced System Settings:
	\begin{verbatim}
		  Ctrl+r > Fill in "control sysdm.cpl" > Press enter to run it
		  > Advanced > Environment Variables
	\end{verbatim}
%	\begin{lstlisting}
%		Ctrl+R --> Fill in "control sysdm.cpl" --> Press enter to run it --> Advanced --> Environment Variables
%	\end{lstlisting}
	% Beter alternatief vinden dan lstlisting

	Here you will see two areas, one are the user variables and the other are the system variables. We will be adding two variables to the last one. Simply press "New..." and enter the following information, substituting <breve\_path> with the actual path on your pc:
	\begin{verbatim}
		  Variable name: BREVE_CLASS_PATH
		  Variable value: <breve_path>\lib\classes
	\end{verbatim}
%	\begin{lstlisting}
%		Variable name: BREVE_CLASS_PATH
%		Variable value: <breve_path>\lib\classes
%	\end{lstlisting}
	\begin{verbatim}
		  Variable name: PYTHONPATH
		  Variable value: <breve_path>\lib\python2.3
	\end{verbatim}
%	\begin{lstlisting}
%		Variable name: PYTHONPATH
%		Variable value: <breve_path>\lib\python2.3
%	\end{lstlisting}

\subsection{Testing}
	Breve comes with several demos. You can use these to check whether the installation went okay. 

	You can run breve by running the breve executable followed by a project, don't forget to set the paths if you are using the temporary version.

	The following command prompt command will run the DLA demo:
	\begin{verbatim}
		  <breve_path>\bin\breve <breve_path>\demos\DLA.py
	\end{verbatim}
%	\begin{lstlisting}
%	<breve_path>\bin\breve <breve_path>\demos\DLA.py
%	\end{lstlisting}
	% Nog toevoegen hoe het zit als je al Python in gebruik hebt en al paden hebt ignesteld

\section{Mac OS X}
	In this section we will explain how to set up Breve under Mac OS X

\subsection{Installation}
	Extract the files to a location of your choosing.

\subsection{Setting path}
	On Mac OS X you only have to set one path. As with Windows you have two options, one is only temporary while the second is sort of permanent. 

	First of all the temporary version. Run the following line each time you start a new terminal session in which you will be using Breve:
	\begin{verbatim}
		  export BREVE_CLASS_PATH=<breve_path>/lib/classes
	\end{verbatim}
%	\begin{lstlisting}
%		export BREVE_CLASS_PATH=<breve_path>/lib/classes
%	\end{lstlisting}

	For the second, more permament version, you will have to change/create a .bash\_profile file. Anything in this file will be run each time you start a terminal session. Run the following lines in a terminal session to create or change the file:
	\begin{verbatim}
		  nano ~/.bash_profile
	\end{verbatim}
%	\begin{lstlisting}
%		nano ~/.bash_profile
%	\end{lstlisting}
	Now you have either created a .bash\_profile file or you are editing the existing version. Enter the following line:
	\begin{verbatim}
		  export BREVE_CLASS_PATH=<breve_path>/lib/classes
	\end{verbatim}
%	\begin{lstlisting}
%		export BREVE_CLASS_PATH=<breve_path>/lib/classes
%	\end{lstlisting}
	Now you can save this file by pressing `ctrl+o' and you can leave this file by pressing `ctrl+x'. After restarting the terminal you should be able to use Breve. 

\subsection{Testing}
	Breve comes with several demos. You can use these to check whether the installation went okay. 

	You can run breve by running the breve executable followed by a project, don't forget to export the path if you are using the temporary version.
	
	The following terminal command will run the DLA demo:
	\begin{verbatim}
		  <breve_path>/bin/breve <breve_path>/demos/DLA.py
	\end{verbatim}
%	\begin{lstlisting}
%	<breve_path>/bin/breve <breve_path>/demos/DLA.py
%	\end{lstlisting}
	%bin/breve demos/gatherers.tz
