\section{Adding interaction between food and agents}
Having food that does nothing is not interesting, we want to be able to interact with it. When an agent touches a food source he should pick it up and carry it for a while, before dropping it again.

In our case a food source can only be pushed by one agent. So let us start with that. We will have to add an owner to the food to see if it is being pushed.

\begin{lstlisting}
self.owner = None
\end{lstlisting}

\instruct{Add the line initializing the owner to the \textit{\_\_init\_\_} function of the food}
Now we want to be able to get and set this value from other classes. This can be done by adding a getter and a setter.
\begin{lstlisting}
def setOwner(self, o):
    self.owner = o

def getOwner(self):
    return self.owner
\end{lstlisting}
\instruct{Add the two functions to the food file}
That was all we had to change to the food file. From now on we will be working in the agent file. First we will add whether an agent is carrying an object or not.
\begin{lstlisting}
# Store a possible food source
self.carrying = None
\end{lstlisting}

\instruct{Add the line initializing the carrying status to the \textit{\_\_init\_\_} function of the agent}
First we will define what happens if an agent collides with a food source. There are several possibilities. The food you collide with could already be carried by someone or you have recently collided with another food source, in these two cases you do nothing. If you are carrying food and you collide with a food source that no one is carrying, you place your current food on the ground. If you are not carrying food and none of the above possibilities apply, you will pick the food up.
\begin{lstlisting}
def collisionWithFood(self, f):
    # Checks if the food is already being carried
    if(f.getOwner()):
        return

    # We want two iterations without an collision
    if(self.collidedTimer > 0):
        self.collidedTimer = 2

    # Set the timer for the collision
    self.collidedTimer = 2

    if(self.carrying != None):
        self.placeFoodObject(self.carrying, f)
        self.carrying = None
        return

    self.carrying = f
    self.carrying.setOwner(self)
\end{lstlisting}

\instruct{Add the collion function to the RandomAgent class}
You might have noticed that we talked about placing food, something we will implement now. The food you are carrying is placed near the food you collided with.

\begin{fullwidth}
\begin{lstlisting}
def placeFoodObject(self, ownFood, placedFood):

    location = placedFood.getLocation()
    location = location + ( breve.randomExpression( breve.vector( 2, 2, 2 ) ) - breve.vector( 1, 1, 1 ) )

    ownFood.move(location)
    ownFood.setOwner(0)
\end{lstlisting}
\end{fullwidth}

\instruct{Add the placeFoodObject function above to the RandomAgent class}
In the collision function we used a timer which represents how long you have not touched any food. This is so you do not pick up the same object right away. However we have not initialized this variable yet.
\begin{fullwidth}
\begin{lstlisting}
# Timer to prevent / to inhibit the collision behavior
self.collidedTimer = 0
\end{lstlisting}
\end{fullwidth}

\instruct{Add the line initializing the timer to the \textit{\_\_init\_\_} function of the agent}
Changing the timer will happen in the iterator, so that every refresh this value is decreased.

\begin{lstlisting}
self.collidedTimer -= 1
\end{lstlisting}

\instruct{Add the line that decreases the timer to the \textit{iterate} function}
At the moment we have all necessary functions to describe what happens when you collide with food. We are still missing what actually happens to the food while you are carrying it. When you are carrying food the location of the food should be adjusted along with your location.
\begin{fullwidth}
\begin{lstlisting}
if(self.carrying):
    self.carrying.move((self.getLocation() - breve.vector(1,0,0)))
\end{lstlisting}
\end{fullwidth}

\instruct{Add the if and its body to the \textit{\_\_init\_\_} function of the agent}
We now have all necessary functions to interact with and carry food. But our agents do not know yet when they actually collide with food. For this we will use a standard breve function. First you give the object that it is colliding with and next you give the function that should be executed in case of collision.
\begin{fullwidth}
\begin{lstlisting}
# Setup the collision
self.handleCollisions('SimpleFood', 'collisionWithFood')
\end{lstlisting}
\end{fullwidth}
\instruct{Add the breve function and its parameters to the \textit{\_\_init\_\_} function of the agent}

