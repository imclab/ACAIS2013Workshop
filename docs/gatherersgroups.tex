\section{Creating two groups: Red versus Blue}
Now that we have finished creating a basic Gatherers simulation, it is time to go one step further. When creating an Artificial Life simulation, most of the time you will want to create different conditions of the environment and compare them to each other. We won't really make different conditions, but we will create two groups of agents who will compete with each other in the same world: the \textit{Red Agents} and the \textit{Blue Agents}.

First, we will create two new classes, \textit{RedAgent}  and \textit{BlueAgent} underneath our RandomAgent. These new types of agent will extend our RandomAgent. Then we need a property \textit{group} to indicate to which group they belong. We will also give them a color so we can distinguish them from each other ourself when we watch the simulation. The color can be specified with the setColor(r, g, b) function where r, g, b are values between 0 and 1 for \textit{red, green} and \textit{blue} respectively. 

These extra properties should be set in the constructor. Naturally, these classes will also need an iterate function. However, these will do nothing special so simply calling the iterate function of the superclass is sufficient. 

\instruct{Add the RedAgent and BlueAgent classes below the RandomAgent class}

\begin{lstlisting}[language=Python]
class BlueAgent (RandomAgent):
	def __init__(self):
		RandomAgent.__init__(self)
		self.setColor(breve.vector(0.2,0.2,0.8))
		self.group = 1

	def iterate(self):
		RandomAgent.iterate(self)


class RedAgent (RandomAgent):
	def __init__(self):
		RandomAgent.__init__(self)
		self.setColor(breve.vector(0.8,0.2,0.2))
		self.group = 2

	def iterate(self):
		RandomAgent.iterate(self)
\end{lstlisting}

We will also want to give the RandomAgent a default group it belongs, just to make sure that we will not brake the interaction between food and agents in a bit.

\instruct{In the constructor of RandomAgent, add a line stating that it belongs to group 0}

\begin{lstlisting}[language=Python]
self.group = 0
\end{lstlisting}

Now that we created two new classes for the two groups of agents, we will have to make sure they are actually going to compete with each other (by stealing each other's food). We can do this by taking into account the group an agent belongs to when it interacts with a food source. 

The first step is to mark the food that an agent picks up as belonging to the group of that agent. In order to do this we must first update the \textit{SimpleFood} class so that it supports belonging to a group. We do this by adding a property \textit{group} in the constructor, just like we did for the agents. As all food starts off as neutral, it will belong to group 0. 

\instruct{In the constructor of SimpleFood, add a line stating that it belongs to group 0 (it is still neutral)}

\begin{lstlisting}[language=Python]
self.group = 0
\end{lstlisting}

Of course, we will also want to be able to check to what group food belongs to and change it. Thus, we must create a getter and setter function for the group-property.

\instruct{Add a getter and setter for the group-property at the bottom of the class}

\begin{lstlisting}[language=Python]
def setGroup(self, o):
	self.group = o

def getGroup(self):
	return self.group
\end{lstlisting}

Now we should make use of this property by making agents state that the food they pickup belongs to their group. We can do this by updating the \textit{collissionWithFood()} function you created and changing the group-property of the food source with the setter we just created. While we are doing that, we will change the color of the food source as well to be the same as the color of the agent so we can see what is happening in the simulation.

\instruct{Update the RandomAgent's collisionWithFood() function to update the group and color of a food source. Add the following lines to the bottom of the function}

\begin{lstlisting}[language=Python]
		self.carrying.setColor(self.getColor())
		self.carrying.setGroup(self.group)
\end{lstlisting}

Next up, we want to make sure that agents only deposit the food they are carrying near either neutral food or food of their group has already claimed, since they definitely do not want to put it down near the stack of the other group!

To do this, we must once again update the collisionWithFood() function and check to what group the newly encountered food belongs to. If it is neutral food or belongs to the agent's group, the food should be dropped down and the agent can continue on its way. If it belongs to the other group, the agent should hold on to his food and ignore the newly encountered food source. We can do this by simply making the function return.

\instruct{Update the RandomAgent's collisionWithFood() function by adding the follow snippet directly below the if(self.carrying != None):-statement:}

\begin{lstlisting}[language=Python]
if(self.carrying.getGroup() != f.getGroup() and f.getGroup() != 0):
	return
\end{lstlisting}


That's it! That's all there is needed in order to create two competing groups of agents that steal each other's food and only put down the food they are holding when they encounter neutral food or food already belonging to their group!

The last thing to do is updating the SimulationController by removing the \textit{breve.createInstances(agents.RandomAgent, 10)}  line and replacing it with two lines that create Red and Blue agents. 

\instruct{Update the SimulationController by creating Red and Blue agents:}

\begin{lstlisting}[language=Python]
breve.createInstances(agents.RedAgent, 10)
breve.createInstances(agents.BlueAgent, 10)
\end{lstlisting}

Now run the simulation and watch the agents claim and gather food for their groups.